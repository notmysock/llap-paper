\section{Hybrid Execution}

LLAP does not take up the responsibility of performing task orchestration or determining where each task should run on. 
This however is not possible directly by an external DAG engine such as Apache Tez\cite{tez}, without having access to
a common vocabulary for interaction. LLAP from the orchestration engine's perspective is another implementation of the
ContainerLauncher, but in this case one which talks to LLAP instance over rpc instead of the YARN NodeManager. Hive and
Tez together come up with a DAG plan, which is then divided into many query fragments which are self-contained and can 
execute on any LLAP instance, but has a definite preference towards cache affinity and HDFS data locality.

The fragments are scheduled onto LLAP daemons published under a name in the service registry. For the sake of enabling 
Hybrid execution mode, the events streaming out of LLAP are exactly identical to the events obtained from the Tez containers
and handled by a plugin service loaded into Tez ApplicationMaster. LLAP however receives some special events from the
plugin when communicating downstream, since there are special indicators for clean up operations for queries which have
completed.

The query fragments kick off the IO elevator and initialize the operator pipeline in parallel. In some cases, the fragments
might discover that other fragments which require similar initialization are running in the same process and decide to 
co-operate on expensive initialization operations like building broadcast-join hashtables, which will be built exactly once
per node instead of being build multiple times in each fragment. If a query fragment happens to be waiting on a machine the
fragment will send heartbeats out eventhough it is not running, to let the DAG master know that the process hosting the
fragment is still alive.

The query fragments are the basic unit of failure tolerance. Whenever a query fragment fails, it gets rescheduled until it
has failed for the maximum number of times, the process of rescheduling it can even involve terminating a running fragment
which is part of the same query depending on the priority between the fragments. These operations within a single DAG execution
maps exactly to the dead lock prevention already in the Tez vertex manager plugins, differing only in the means of communication -
however, the speed at which LLAP fragments execute result in far higher likelihood of exposing race conditions within future 
vertexmanager plugins.


In secure clusters, the orchestration \& discovery can only happen with a single user, to avoid accidental security leakage
between users inhabiting the same process and sharing the same cache. This is not as much hardship as it sounds as the 
Apache Ranger\cite{ranger} perimeter security translates all users into a single user after security verifications are 
complete, to ensure that no user has actual file level access to tables.
