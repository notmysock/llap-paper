\section{Background}

LLAP is the most recent addition in a long journey Hive has taken as part of the Stinger and Stinger.next initiatives, which has added 
Apache Tez \cite{tez}, Apache ORC \cite{orc} and Apache Calcite \cite{cbo} to Hive's repertoire.
LLAP is unique in its pursuit of the low-latency goals as it does not try to reinvent a completely new SQL engine. It offers a column service 
layer which interacts with the Hive execution engine, with dynamic decisions on whether a query or even parts of a query render down 
into the LLAP query fragments. As a flexible optional accelerator, this brings it closer to production readiness than other solutions
which require an all-or-nothing adoption. 

LLAP is composed of the eponymous long lived distributed processes, which offers temporary 
control over compute scheduling and data placement guarantees during the execution of a short running 
query. These processes remove the restrictions and overheads imposed by a large scale resource manager
such as YARN\cite{YARN} which allocate requested resources on a multi-second heartbeat and only offer
pre-emption in the scale of tens of seconds.

LLAP as a long-lived column service is the central pillar on which all of the subsequent improvements
rest upon. 
