\section{Experiments \& Takeaways}

Before dissecting the internals of LLAP, we take a brief digression into deployments and experiments, 
which add colour to the architectural discussions. 

\subsection{Yahoo Display Network Benchmark}

Yahoo Japan published a timeseries reporting business case as part of a benchmark\cite{impala_perf} comparing
latencies of Cloudera Impala\cite{impala} against Apache Hive. This use-case was based on a production scenario
which features a 500 billion row warehouse which has significant skews in its data distribution. The challenge 
described in the use-case stemmed from having both throughput and latency SLAs and simulated concurrent users
creating extracted reports for further analysis.

The use-case divided produced aggregate of metrics bounded by segments, of selected attributes. The segments 
were primarily the start of the period and the length of the period - daily, weekly, monthly and yearly. The 
attributes were selection criteria which included user demographics and object identifiers. This does not 
easily map into a pre-aggregate MOLAP system as the total number of object identifiers were very large and 
the cross-product against user demographics is only a ~6x reduction in cardinality, which was negated by having
4 periods of aggregation.

The query pattern for the time-series had a skew towards the last day and last week, which affects a MPP share-nothing
model adversely by producing a CPU hot-spot around the machines holding the last billion rows. Eventhough Tez was not
affected by that specific issue due to locality delay scheduling, it was still producing secondary network hotspots
around the storage nodes which contained more popular user-demographics. 

This experiment gave valuable feedback into the design of LLAP. Along with use of Tez sessions, the benchmark 
reached an acceptable 5.67 queries per second (20412 queries per hour, exceeding the 15000 queries per hour SLA).

As part of validating LLAP's solution, the exact same benchmark was setup with the latest Hive and run through for the following
query patterns with 20 users, each user getting a unique tuple of parameters.

\begin{itemize}
\item \textbf{Query1x}: by object id
\item \textbf{Query2x}: by dayOfWeek
\item \textbf{Query6x}: across 50 object ids by device
\item \textbf{Query7x}: across 50 object ids by dayOfWeek 
\end{itemize}

\begin{table}[h]
\begin{tabular}{l|*{4}c}
Query1x  &   Tez (avg)  &   Tez (worst)  &   LLAP (avg)  &   LLAP (worst) \\
\hline
One Day  &   2.53  &   4.5  &   1.33  &   3 \\
One Week  &   3  &   6.18  &   1.37  &   2.9 \\
One Month  &   4  &   7.28  &   1.35  &   2.7 \\
One year  &   7.4  &   19.51  &   3  &   17.22 \\
\end{tabular}
\caption{Query1x across different periods (s)}
\end{table}

\begin{table}[h]
\begin{tabular}{l|*{4}c}
Query2x  &   Tez (avg)  &   Tez (worst)  &   LLAP (avg)  &   LLAP (worst) \\
\hline \\
One Day  &   2.556  &   6.67  &   1.11  &   1.74 \\
One Week  &   1.99  &   2.84  &   1.01  &   1.47 \\
One Month  &   3.07  &   6.65  &   1.29  &   2.2 \\
One year  &   5.03  &   7.47  &   1.74  &   4.67 \\
\end{tabular}
\caption{Query2x across different periods (s)}
\end{table}

\begin{table}[h]
\begin{tabular}{l|*{4}c}
Query6x  &   Tez (avg)  &   Tez (worst)  &   LLAP (avg)  &   LLAP (worst) \\
\hline \\
One Day  &   1.9  &   2.6  &   1.17  &   3.63 \\
One Week  &   2.22  &   5.9  &   1.03  &   1.66 \\
One Month  &   3.5  &   6.7  &   1.23  &   2.76 \\
One year  &   6  &   7.27  &   2  &   5.44 \\
\end{tabular}
\caption{Query6x across different periods (s)}
\end{table}

\begin{table}[h]
\begin{tabular}{l|*{4}c}
Query7x  &   Tez (avg)  &   Tez (worst)  &   LLAP (avg)  &   LLAP (worst) \\
\hline \\
One Day  &   1.9  &   2.6  &   1.17  &   3.63 \\
One Week  &   2.22  &   5.9  &   1.03  &   1.66 \\
One Month  &   3.5  &   6.7  &   1.23  &   2.76 \\
One year  &   6  &   7.27  &   2  &   5.44 \\
\end{tabular}
\caption{Query7x across different periods (s)}
\end{table}

LLAP managed to lower the latency as well as hit a far higher throughput of 15.16 q/s (54576 queries per hour). The 
performance bottlenecks with LLAP had moved from the data skew towards the cpu utilized by the bottlenecks within
the JDBC layer in HiveServer2. This was isolated from the use-case by increasing the total number of users in the system and
increasing the number of HiveServer2 until worst case latencies dropped out of SLA.

\begin{table}[h]
\begin{tabular}{l|*{3}c}
Concurrency & Q/S & LLAP(avg) & LLAP (worst) \\
\hline \\
5 Users x 1 HS2 & 4.51 & 1.11 & 1.91 \\
10 Users x 1 HS2 & 8.67 & 1.15 & 1.8 \\
20 Users x 1 HS2 & 15.16 & 1.32 & 2.26 \\
20 Users x 3 HS2 & 14.71 & 1.36 & 2.13 \\
48 Users x 3 HS2 & 25.07 & 1.92 & 3.94 \\
72 Users x 3 HS2 & 30.62 & 2.35 & 5.64 \\
96 Users x 3 HS2 & 34.19 & 2.81 & 7.39 \\
\end{tabular}
\caption{HiveServer2 Concurrency Tests}
\end{table}

\subsection{Interactive Queries}

The TPC-DS data-set is a better candiate for a repeatable benchmark for LLAP, allowing any cluster to generate
identical data at scale factors for testing. The experimental setup is a SCALE=1000 TPC-DS data set using the 
schema from hive-testbench\cite{testbench}, with interactive queries 
27, 42, 52, 55, 68, 73 and 98.

\begin{table}[h]
\begin{tabular}{l|*{1}cr}
Query & LLAP time (s) & Vertices \\
\hline \\
Query 27 & 4.35 & 7  \\
Query 42 & 0.97 & 5  \\
Query 52 & 1.27 & 5  \\
Query 55 & 0.89 & 5 \\
Query 68 & 2.86 & 9 \\
Query 73 & 1.53 & 7 \\
Query 98 & 3.90 & 6 \\
\end{tablular}
\caption{Interactive Queries on LLAP}
\end{table}

In this example, the compilation time for the query has been excluded from the benchmark as it overshadows the
total runtime of the query.

\subsection{Index Caching}

Lookups for a single primary key can populate indexes for the primary key very rapidly 
across the cluster, so that further primary key lookups are also accelerated. 
The TPC-H data-set is another standardized benchmark data-set which is useful to demonstrate the impact of 
LLAP's index fragment caching, with a SCALE=1000 dataset which is loaded with ORC bloom filters.

\begin{lstlisting}[language=sql]
select count(1) from lineitem where l_orderkey=?? ;
\end{lstlisting}

\begin{table}[h]
\begin{tabular}{l|*{3}cr}
Query & l_orderkey=2 & l_orderkey=3 & l_orderkey=567\\
\hline \\
LLAP times & 3.06 & 0.93 & 1.01 \\
Tez times & 7.88 & 7.65 & 9.40  \\
\end{tablular}
\caption{Index Caching}
\end{table}

The cold run is slower, but the warm runs use the bloom filter indexes which were loaded for the previous run, while Tez has no such 
caching for bloom filter indexes.

%% \subsection{Cloud Storage: Azure}

%% LLAP: 128 nodes + non-local data

%% Q1 baseline is 357s -> 50s at 1Tb scale.

